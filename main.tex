%%%%%%%%%%%%%%%%%%%%%%%%%%%%%%
% 1-Page Tech CV - Yang Sen (Liam) Lin
%%%%%%%%%%%%%%%%%%%%%%%%%%%%%%
\documentclass[10pt,letterpaper]{article}
\usepackage[margin=0.7in]{geometry}
\usepackage[hidelinks]{hyperref}
\usepackage{enumitem}
\usepackage{titlesec}
\usepackage{setspace}
\usepackage{array}
\usepackage{tabularx}
\usepackage{ragged2e}
\usepackage{fontawesome5}
\setstretch{1.04}

% Section headings
\titleformat{\section}{\large\bfseries}{}{0em}{}[\vspace{2pt}\titlerule]
\titlespacing*{\section}{0pt}{6pt}{4pt}

% Lists
\setlist[itemize]{
  leftmargin=1.6em,
  labelsep=0.4em,
  topsep=2pt,
  itemsep=2pt,
  parsep=0pt
}
% Hanging bullet environment
\newenvironment{hangitems}{%
  \begin{itemize}[leftmargin=1.6em,labelsep=0.4em,topsep=2pt,itemsep=2pt,parsep=0pt]
    \setlength{\parindent}{0pt}%
}{%
  \end{itemize}
}

% Entry helpers
\newcommand{\entry}[3]{%
  \noindent\textbf{#1} \hfill #2%
  \if\relax\detokenize{#3}\relax
  \else
    \\[-1pt]\textit{#3}\par
  \fi
  \vspace{2pt}%
}
\newcommand{\sep}{\;\textbar\;}
\newcolumntype{Y}{>{\RaggedRight\arraybackslash}X}
\newcolumntype{D}{>{\raggedleft\arraybackslash}p{31mm}}
\newcommand{\daterange}[2]{%
  {\small\textit{#1\,--\,#2}}%
}

\begin{document}
%========== Header ==========
\begin{center}
  {\Large \bfseries Yang Sen (Liam) Lin}\\[4pt]
  \faIcon{envelope} \ \href{mailto:tommy60718.en10@nycu.edu.tw}{tommy60718.en10@nycu.edu.tw}
  \quad\sep\quad
  \faIcon{github} \ \href{https://github.com/tommy60718}{github.com/tommy60718}
  \quad\sep\quad
  \faIcon{linkedin} \ \href{https://www.linkedin.com/in/}{linkedin.com/in/}
\end{center}

%========== Education ==========
\section*{Education}
\textbf{National Yang Ming Chiao Tung University (NYCU)} \hfill Sep 2021 -- Dec 2025 (expected)\\
B.S., Mechanical Engineering \& Computer Science (cross-disciplinary) \hfill GPA: 4.04/4.3 (last 60), 3.85/4.3 (overall)

%========== Publications ==========
\section*{Publications}
\textbf{\href{https://spf-web.pages.dev/}{See, Point, Fly — a training-free vision-language UAV navigation framework}. CoRL 2025}\\[-1pt]
Chih Yao Hu*, \textbf{Yang-Sen Lin}*, Yuna Lee, Chih-Hai Su, Jie-Ying Lee, Shr-Ruei Tsai, Chin-Yang Lin, Kuan-Wen Chen, Tsung-Wei Ke, Yu-Lun Liu
\\[-2pt]\small{* equal contribution}\normalsize

%========== Work & Research ==========
\section*{Work \& Research Experience}
\entry{NYCU Computational Photography Lab}{Sep 2024 -- Present}{Undergraduate Researcher \hfill Hsinchu, Taiwan}
\begin{hangitems}
  \item Research skills: paper survey and reimplementation, paper writing (\href{https://spf-web.pages.dev/}{See, Point, Fly}), Spatial math problem-solving.
  \item Engineering skills: simulator-to-real-world system design, experiment design, and evaluation design.
\end{hangitems}

\entry{Wolley Inc.}{Jul 2024 -- Aug 2024}{Firmware Engineer Intern \hfill Hsinchu, Taiwan}
\begin{hangitems}
  \item Optimized CXL Type-3 (HDM) integration; reduced host bandwidth pressure; improved I/O efficiency.
  \item Built C diagnostics/profiling for throughput/latency.
\end{hangitems}

\entry{Google Developer Student Club (GDSC) NYCU}{Jul 2024 -- Jun 2025}{Organization Lead \hfill Hsinchu, Taiwan (Hybrid)}
\begin{hangitems}
  \item Led GDSC NYCU for one year, managing a team of 24 members across five departments. Oversaw 6 AI/Software Engineering project teams and organized over 10 technical events, including software development workshops and technical sharing sessions.
\end{hangitems}

%========== Projects ==========
\section*{Projects}
\renewcommand{\arraystretch}{1.0} % Tightened from 1.05 for less vertical space
\begin{tabularx}{\linewidth}{@{}Y D@{}}
  \textbf{\href{https://github.com/Hu-chih-yao/see-point-fly}{See, Point, Fly — Learning-Free VLM UAV Navigation}} & \daterange{Nov~2024}{Aug~2025} \\
  Zero-shot language-guided UAV control. See, Point, Fly (SPF) enables UAVs to navigate to any goal based on free-form natural language instructions in any environment, without task-specific training. \textit{Skills:} Python, mss, Matplotlib, VLM. & \\
  \textbf{\href{https://github.com/tommy60718/2023_UAV-CV}{Vision-based UAV Autopilot}} & \daterange{Sep~2023}{Dec~2023} \\
  Build autonomous Tello drone: detect ArUco, navigate marker course, follow black line without backtracking, and land precisely on final marker. \textit{Skills:} Python, OpenCV, NumPy, djitellopy (Tello), PID control, camera calibration, morphology/edges, YOLOv7 (GPU). & \\
  \textbf{\href{https://github.com/tommy60718/2023_flappy_bird}{C++ Flappy Bird (SDL)}} & \daterange{Sep~2023}{Nov~2023} \\
  SDL graphics/audio; physics+collision; jitter fix via \texttt{SDL\_GetTicks()}; latency 50\%$\rightarrow$\textless{}5\%; OOP refactor — \textit{Skills:} C++, SDL, game loop, OOP & \\
\end{tabularx}

%========== Awards & Extracurricular ==========
\section*{Awards \& Extracurricular}
\begin{itemize}[leftmargin=1.2em,itemsep=2pt,topsep=2pt]
  \item \textbf{Academic Achievements Award} — Top 5\% (NYCU) \hfill 2023 Fall
  \item \textbf{Academic Achievements Award} — Top 5\% (NYCU) \hfill 2024 Fall
  \item \textbf{Taipei Metro Hackathon} — 2nd Place \& Popularity Award (among 92 teams) \hfill May 2024
  \item \textit{Leadership/Activities:} Swimming Team (3y), GDSC Lead (1y), Guitar Club Instructor (1y)
\end{itemize}

%========== Skills ==========
\section*{Skills}
\textbf{Languages:} Python, C/C++ \qquad
\textbf{AI/ML/Robotics:} OpenCV, YOLO, VLMs, control (PID) \\
\textbf{Systems/Firmware:} CXL Type-3, HDM concepts, embedded profiling \qquad
\textbf{Tools:} Linux, Git, SDL \\
\textbf{Strengths:} Algorithm optimization, computer vision, robotics programming, research, team leadership

\end{document}